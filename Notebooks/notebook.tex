
% Default to the notebook output style

    


% Inherit from the specified cell style.




    
\documentclass[11pt]{article}

    
    
    \usepackage[T1]{fontenc}
    % Nicer default font (+ math font) than Computer Modern for most use cases
    \usepackage{mathpazo}

    % Basic figure setup, for now with no caption control since it's done
    % automatically by Pandoc (which extracts ![](path) syntax from Markdown).
    \usepackage{graphicx}
    % We will generate all images so they have a width \maxwidth. This means
    % that they will get their normal width if they fit onto the page, but
    % are scaled down if they would overflow the margins.
    \makeatletter
    \def\maxwidth{\ifdim\Gin@nat@width>\linewidth\linewidth
    \else\Gin@nat@width\fi}
    \makeatother
    \let\Oldincludegraphics\includegraphics
    % Set max figure width to be 80% of text width, for now hardcoded.
    \renewcommand{\includegraphics}[1]{\Oldincludegraphics[width=.8\maxwidth]{#1}}
    % Ensure that by default, figures have no caption (until we provide a
    % proper Figure object with a Caption API and a way to capture that
    % in the conversion process - todo).
    \usepackage{caption}
    \DeclareCaptionLabelFormat{nolabel}{}
    \captionsetup{labelformat=nolabel}

    \usepackage{adjustbox} % Used to constrain images to a maximum size 
    \usepackage{xcolor} % Allow colors to be defined
    \usepackage{enumerate} % Needed for markdown enumerations to work
    \usepackage{geometry} % Used to adjust the document margins
    \usepackage{amsmath} % Equations
    \usepackage{amssymb} % Equations
    \usepackage{textcomp} % defines textquotesingle
    % Hack from http://tex.stackexchange.com/a/47451/13684:
    \AtBeginDocument{%
        \def\PYZsq{\textquotesingle}% Upright quotes in Pygmentized code
    }
    \usepackage{upquote} % Upright quotes for verbatim code
    \usepackage{eurosym} % defines \euro
    \usepackage[mathletters]{ucs} % Extended unicode (utf-8) support
    \usepackage[utf8x]{inputenc} % Allow utf-8 characters in the tex document
    \usepackage{fancyvrb} % verbatim replacement that allows latex
    \usepackage{grffile} % extends the file name processing of package graphics 
                         % to support a larger range 
    % The hyperref package gives us a pdf with properly built
    % internal navigation ('pdf bookmarks' for the table of contents,
    % internal cross-reference links, web links for URLs, etc.)
    \usepackage{hyperref}
    \usepackage{longtable} % longtable support required by pandoc >1.10
    \usepackage{booktabs}  % table support for pandoc > 1.12.2
    \usepackage[inline]{enumitem} % IRkernel/repr support (it uses the enumerate* environment)
    \usepackage[normalem]{ulem} % ulem is needed to support strikethroughs (\sout)
                                % normalem makes italics be italics, not underlines
    

    
    
    % Colors for the hyperref package
    \definecolor{urlcolor}{rgb}{0,.145,.698}
    \definecolor{linkcolor}{rgb}{.71,0.21,0.01}
    \definecolor{citecolor}{rgb}{.12,.54,.11}

    % ANSI colors
    \definecolor{ansi-black}{HTML}{3E424D}
    \definecolor{ansi-black-intense}{HTML}{282C36}
    \definecolor{ansi-red}{HTML}{E75C58}
    \definecolor{ansi-red-intense}{HTML}{B22B31}
    \definecolor{ansi-green}{HTML}{00A250}
    \definecolor{ansi-green-intense}{HTML}{007427}
    \definecolor{ansi-yellow}{HTML}{DDB62B}
    \definecolor{ansi-yellow-intense}{HTML}{B27D12}
    \definecolor{ansi-blue}{HTML}{208FFB}
    \definecolor{ansi-blue-intense}{HTML}{0065CA}
    \definecolor{ansi-magenta}{HTML}{D160C4}
    \definecolor{ansi-magenta-intense}{HTML}{A03196}
    \definecolor{ansi-cyan}{HTML}{60C6C8}
    \definecolor{ansi-cyan-intense}{HTML}{258F8F}
    \definecolor{ansi-white}{HTML}{C5C1B4}
    \definecolor{ansi-white-intense}{HTML}{A1A6B2}

    % commands and environments needed by pandoc snippets
    % extracted from the output of `pandoc -s`
    \providecommand{\tightlist}{%
      \setlength{\itemsep}{0pt}\setlength{\parskip}{0pt}}
    \DefineVerbatimEnvironment{Highlighting}{Verbatim}{commandchars=\\\{\}}
    % Add ',fontsize=\small' for more characters per line
    \newenvironment{Shaded}{}{}
    \newcommand{\KeywordTok}[1]{\textcolor[rgb]{0.00,0.44,0.13}{\textbf{{#1}}}}
    \newcommand{\DataTypeTok}[1]{\textcolor[rgb]{0.56,0.13,0.00}{{#1}}}
    \newcommand{\DecValTok}[1]{\textcolor[rgb]{0.25,0.63,0.44}{{#1}}}
    \newcommand{\BaseNTok}[1]{\textcolor[rgb]{0.25,0.63,0.44}{{#1}}}
    \newcommand{\FloatTok}[1]{\textcolor[rgb]{0.25,0.63,0.44}{{#1}}}
    \newcommand{\CharTok}[1]{\textcolor[rgb]{0.25,0.44,0.63}{{#1}}}
    \newcommand{\StringTok}[1]{\textcolor[rgb]{0.25,0.44,0.63}{{#1}}}
    \newcommand{\CommentTok}[1]{\textcolor[rgb]{0.38,0.63,0.69}{\textit{{#1}}}}
    \newcommand{\OtherTok}[1]{\textcolor[rgb]{0.00,0.44,0.13}{{#1}}}
    \newcommand{\AlertTok}[1]{\textcolor[rgb]{1.00,0.00,0.00}{\textbf{{#1}}}}
    \newcommand{\FunctionTok}[1]{\textcolor[rgb]{0.02,0.16,0.49}{{#1}}}
    \newcommand{\RegionMarkerTok}[1]{{#1}}
    \newcommand{\ErrorTok}[1]{\textcolor[rgb]{1.00,0.00,0.00}{\textbf{{#1}}}}
    \newcommand{\NormalTok}[1]{{#1}}
    
    % Additional commands for more recent versions of Pandoc
    \newcommand{\ConstantTok}[1]{\textcolor[rgb]{0.53,0.00,0.00}{{#1}}}
    \newcommand{\SpecialCharTok}[1]{\textcolor[rgb]{0.25,0.44,0.63}{{#1}}}
    \newcommand{\VerbatimStringTok}[1]{\textcolor[rgb]{0.25,0.44,0.63}{{#1}}}
    \newcommand{\SpecialStringTok}[1]{\textcolor[rgb]{0.73,0.40,0.53}{{#1}}}
    \newcommand{\ImportTok}[1]{{#1}}
    \newcommand{\DocumentationTok}[1]{\textcolor[rgb]{0.73,0.13,0.13}{\textit{{#1}}}}
    \newcommand{\AnnotationTok}[1]{\textcolor[rgb]{0.38,0.63,0.69}{\textbf{\textit{{#1}}}}}
    \newcommand{\CommentVarTok}[1]{\textcolor[rgb]{0.38,0.63,0.69}{\textbf{\textit{{#1}}}}}
    \newcommand{\VariableTok}[1]{\textcolor[rgb]{0.10,0.09,0.49}{{#1}}}
    \newcommand{\ControlFlowTok}[1]{\textcolor[rgb]{0.00,0.44,0.13}{\textbf{{#1}}}}
    \newcommand{\OperatorTok}[1]{\textcolor[rgb]{0.40,0.40,0.40}{{#1}}}
    \newcommand{\BuiltInTok}[1]{{#1}}
    \newcommand{\ExtensionTok}[1]{{#1}}
    \newcommand{\PreprocessorTok}[1]{\textcolor[rgb]{0.74,0.48,0.00}{{#1}}}
    \newcommand{\AttributeTok}[1]{\textcolor[rgb]{0.49,0.56,0.16}{{#1}}}
    \newcommand{\InformationTok}[1]{\textcolor[rgb]{0.38,0.63,0.69}{\textbf{\textit{{#1}}}}}
    \newcommand{\WarningTok}[1]{\textcolor[rgb]{0.38,0.63,0.69}{\textbf{\textit{{#1}}}}}
    
    
    % Define a nice break command that doesn't care if a line doesn't already
    % exist.
    \def\br{\hspace*{\fill} \\* }
    % Math Jax compatability definitions
    \def\gt{>}
    \def\lt{<}
    % Document parameters
    \title{1 Data Exploration and Visualisation}
    
    
    

    % Pygments definitions
    
\makeatletter
\def\PY@reset{\let\PY@it=\relax \let\PY@bf=\relax%
    \let\PY@ul=\relax \let\PY@tc=\relax%
    \let\PY@bc=\relax \let\PY@ff=\relax}
\def\PY@tok#1{\csname PY@tok@#1\endcsname}
\def\PY@toks#1+{\ifx\relax#1\empty\else%
    \PY@tok{#1}\expandafter\PY@toks\fi}
\def\PY@do#1{\PY@bc{\PY@tc{\PY@ul{%
    \PY@it{\PY@bf{\PY@ff{#1}}}}}}}
\def\PY#1#2{\PY@reset\PY@toks#1+\relax+\PY@do{#2}}

\expandafter\def\csname PY@tok@w\endcsname{\def\PY@tc##1{\textcolor[rgb]{0.73,0.73,0.73}{##1}}}
\expandafter\def\csname PY@tok@c\endcsname{\let\PY@it=\textit\def\PY@tc##1{\textcolor[rgb]{0.25,0.50,0.50}{##1}}}
\expandafter\def\csname PY@tok@cp\endcsname{\def\PY@tc##1{\textcolor[rgb]{0.74,0.48,0.00}{##1}}}
\expandafter\def\csname PY@tok@k\endcsname{\let\PY@bf=\textbf\def\PY@tc##1{\textcolor[rgb]{0.00,0.50,0.00}{##1}}}
\expandafter\def\csname PY@tok@kp\endcsname{\def\PY@tc##1{\textcolor[rgb]{0.00,0.50,0.00}{##1}}}
\expandafter\def\csname PY@tok@kt\endcsname{\def\PY@tc##1{\textcolor[rgb]{0.69,0.00,0.25}{##1}}}
\expandafter\def\csname PY@tok@o\endcsname{\def\PY@tc##1{\textcolor[rgb]{0.40,0.40,0.40}{##1}}}
\expandafter\def\csname PY@tok@ow\endcsname{\let\PY@bf=\textbf\def\PY@tc##1{\textcolor[rgb]{0.67,0.13,1.00}{##1}}}
\expandafter\def\csname PY@tok@nb\endcsname{\def\PY@tc##1{\textcolor[rgb]{0.00,0.50,0.00}{##1}}}
\expandafter\def\csname PY@tok@nf\endcsname{\def\PY@tc##1{\textcolor[rgb]{0.00,0.00,1.00}{##1}}}
\expandafter\def\csname PY@tok@nc\endcsname{\let\PY@bf=\textbf\def\PY@tc##1{\textcolor[rgb]{0.00,0.00,1.00}{##1}}}
\expandafter\def\csname PY@tok@nn\endcsname{\let\PY@bf=\textbf\def\PY@tc##1{\textcolor[rgb]{0.00,0.00,1.00}{##1}}}
\expandafter\def\csname PY@tok@ne\endcsname{\let\PY@bf=\textbf\def\PY@tc##1{\textcolor[rgb]{0.82,0.25,0.23}{##1}}}
\expandafter\def\csname PY@tok@nv\endcsname{\def\PY@tc##1{\textcolor[rgb]{0.10,0.09,0.49}{##1}}}
\expandafter\def\csname PY@tok@no\endcsname{\def\PY@tc##1{\textcolor[rgb]{0.53,0.00,0.00}{##1}}}
\expandafter\def\csname PY@tok@nl\endcsname{\def\PY@tc##1{\textcolor[rgb]{0.63,0.63,0.00}{##1}}}
\expandafter\def\csname PY@tok@ni\endcsname{\let\PY@bf=\textbf\def\PY@tc##1{\textcolor[rgb]{0.60,0.60,0.60}{##1}}}
\expandafter\def\csname PY@tok@na\endcsname{\def\PY@tc##1{\textcolor[rgb]{0.49,0.56,0.16}{##1}}}
\expandafter\def\csname PY@tok@nt\endcsname{\let\PY@bf=\textbf\def\PY@tc##1{\textcolor[rgb]{0.00,0.50,0.00}{##1}}}
\expandafter\def\csname PY@tok@nd\endcsname{\def\PY@tc##1{\textcolor[rgb]{0.67,0.13,1.00}{##1}}}
\expandafter\def\csname PY@tok@s\endcsname{\def\PY@tc##1{\textcolor[rgb]{0.73,0.13,0.13}{##1}}}
\expandafter\def\csname PY@tok@sd\endcsname{\let\PY@it=\textit\def\PY@tc##1{\textcolor[rgb]{0.73,0.13,0.13}{##1}}}
\expandafter\def\csname PY@tok@si\endcsname{\let\PY@bf=\textbf\def\PY@tc##1{\textcolor[rgb]{0.73,0.40,0.53}{##1}}}
\expandafter\def\csname PY@tok@se\endcsname{\let\PY@bf=\textbf\def\PY@tc##1{\textcolor[rgb]{0.73,0.40,0.13}{##1}}}
\expandafter\def\csname PY@tok@sr\endcsname{\def\PY@tc##1{\textcolor[rgb]{0.73,0.40,0.53}{##1}}}
\expandafter\def\csname PY@tok@ss\endcsname{\def\PY@tc##1{\textcolor[rgb]{0.10,0.09,0.49}{##1}}}
\expandafter\def\csname PY@tok@sx\endcsname{\def\PY@tc##1{\textcolor[rgb]{0.00,0.50,0.00}{##1}}}
\expandafter\def\csname PY@tok@m\endcsname{\def\PY@tc##1{\textcolor[rgb]{0.40,0.40,0.40}{##1}}}
\expandafter\def\csname PY@tok@gh\endcsname{\let\PY@bf=\textbf\def\PY@tc##1{\textcolor[rgb]{0.00,0.00,0.50}{##1}}}
\expandafter\def\csname PY@tok@gu\endcsname{\let\PY@bf=\textbf\def\PY@tc##1{\textcolor[rgb]{0.50,0.00,0.50}{##1}}}
\expandafter\def\csname PY@tok@gd\endcsname{\def\PY@tc##1{\textcolor[rgb]{0.63,0.00,0.00}{##1}}}
\expandafter\def\csname PY@tok@gi\endcsname{\def\PY@tc##1{\textcolor[rgb]{0.00,0.63,0.00}{##1}}}
\expandafter\def\csname PY@tok@gr\endcsname{\def\PY@tc##1{\textcolor[rgb]{1.00,0.00,0.00}{##1}}}
\expandafter\def\csname PY@tok@ge\endcsname{\let\PY@it=\textit}
\expandafter\def\csname PY@tok@gs\endcsname{\let\PY@bf=\textbf}
\expandafter\def\csname PY@tok@gp\endcsname{\let\PY@bf=\textbf\def\PY@tc##1{\textcolor[rgb]{0.00,0.00,0.50}{##1}}}
\expandafter\def\csname PY@tok@go\endcsname{\def\PY@tc##1{\textcolor[rgb]{0.53,0.53,0.53}{##1}}}
\expandafter\def\csname PY@tok@gt\endcsname{\def\PY@tc##1{\textcolor[rgb]{0.00,0.27,0.87}{##1}}}
\expandafter\def\csname PY@tok@err\endcsname{\def\PY@bc##1{\setlength{\fboxsep}{0pt}\fcolorbox[rgb]{1.00,0.00,0.00}{1,1,1}{\strut ##1}}}
\expandafter\def\csname PY@tok@kc\endcsname{\let\PY@bf=\textbf\def\PY@tc##1{\textcolor[rgb]{0.00,0.50,0.00}{##1}}}
\expandafter\def\csname PY@tok@kd\endcsname{\let\PY@bf=\textbf\def\PY@tc##1{\textcolor[rgb]{0.00,0.50,0.00}{##1}}}
\expandafter\def\csname PY@tok@kn\endcsname{\let\PY@bf=\textbf\def\PY@tc##1{\textcolor[rgb]{0.00,0.50,0.00}{##1}}}
\expandafter\def\csname PY@tok@kr\endcsname{\let\PY@bf=\textbf\def\PY@tc##1{\textcolor[rgb]{0.00,0.50,0.00}{##1}}}
\expandafter\def\csname PY@tok@bp\endcsname{\def\PY@tc##1{\textcolor[rgb]{0.00,0.50,0.00}{##1}}}
\expandafter\def\csname PY@tok@fm\endcsname{\def\PY@tc##1{\textcolor[rgb]{0.00,0.00,1.00}{##1}}}
\expandafter\def\csname PY@tok@vc\endcsname{\def\PY@tc##1{\textcolor[rgb]{0.10,0.09,0.49}{##1}}}
\expandafter\def\csname PY@tok@vg\endcsname{\def\PY@tc##1{\textcolor[rgb]{0.10,0.09,0.49}{##1}}}
\expandafter\def\csname PY@tok@vi\endcsname{\def\PY@tc##1{\textcolor[rgb]{0.10,0.09,0.49}{##1}}}
\expandafter\def\csname PY@tok@vm\endcsname{\def\PY@tc##1{\textcolor[rgb]{0.10,0.09,0.49}{##1}}}
\expandafter\def\csname PY@tok@sa\endcsname{\def\PY@tc##1{\textcolor[rgb]{0.73,0.13,0.13}{##1}}}
\expandafter\def\csname PY@tok@sb\endcsname{\def\PY@tc##1{\textcolor[rgb]{0.73,0.13,0.13}{##1}}}
\expandafter\def\csname PY@tok@sc\endcsname{\def\PY@tc##1{\textcolor[rgb]{0.73,0.13,0.13}{##1}}}
\expandafter\def\csname PY@tok@dl\endcsname{\def\PY@tc##1{\textcolor[rgb]{0.73,0.13,0.13}{##1}}}
\expandafter\def\csname PY@tok@s2\endcsname{\def\PY@tc##1{\textcolor[rgb]{0.73,0.13,0.13}{##1}}}
\expandafter\def\csname PY@tok@sh\endcsname{\def\PY@tc##1{\textcolor[rgb]{0.73,0.13,0.13}{##1}}}
\expandafter\def\csname PY@tok@s1\endcsname{\def\PY@tc##1{\textcolor[rgb]{0.73,0.13,0.13}{##1}}}
\expandafter\def\csname PY@tok@mb\endcsname{\def\PY@tc##1{\textcolor[rgb]{0.40,0.40,0.40}{##1}}}
\expandafter\def\csname PY@tok@mf\endcsname{\def\PY@tc##1{\textcolor[rgb]{0.40,0.40,0.40}{##1}}}
\expandafter\def\csname PY@tok@mh\endcsname{\def\PY@tc##1{\textcolor[rgb]{0.40,0.40,0.40}{##1}}}
\expandafter\def\csname PY@tok@mi\endcsname{\def\PY@tc##1{\textcolor[rgb]{0.40,0.40,0.40}{##1}}}
\expandafter\def\csname PY@tok@il\endcsname{\def\PY@tc##1{\textcolor[rgb]{0.40,0.40,0.40}{##1}}}
\expandafter\def\csname PY@tok@mo\endcsname{\def\PY@tc##1{\textcolor[rgb]{0.40,0.40,0.40}{##1}}}
\expandafter\def\csname PY@tok@ch\endcsname{\let\PY@it=\textit\def\PY@tc##1{\textcolor[rgb]{0.25,0.50,0.50}{##1}}}
\expandafter\def\csname PY@tok@cm\endcsname{\let\PY@it=\textit\def\PY@tc##1{\textcolor[rgb]{0.25,0.50,0.50}{##1}}}
\expandafter\def\csname PY@tok@cpf\endcsname{\let\PY@it=\textit\def\PY@tc##1{\textcolor[rgb]{0.25,0.50,0.50}{##1}}}
\expandafter\def\csname PY@tok@c1\endcsname{\let\PY@it=\textit\def\PY@tc##1{\textcolor[rgb]{0.25,0.50,0.50}{##1}}}
\expandafter\def\csname PY@tok@cs\endcsname{\let\PY@it=\textit\def\PY@tc##1{\textcolor[rgb]{0.25,0.50,0.50}{##1}}}

\def\PYZbs{\char`\\}
\def\PYZus{\char`\_}
\def\PYZob{\char`\{}
\def\PYZcb{\char`\}}
\def\PYZca{\char`\^}
\def\PYZam{\char`\&}
\def\PYZlt{\char`\<}
\def\PYZgt{\char`\>}
\def\PYZsh{\char`\#}
\def\PYZpc{\char`\%}
\def\PYZdl{\char`\$}
\def\PYZhy{\char`\-}
\def\PYZsq{\char`\'}
\def\PYZdq{\char`\"}
\def\PYZti{\char`\~}
% for compatibility with earlier versions
\def\PYZat{@}
\def\PYZlb{[}
\def\PYZrb{]}
\makeatother


    % Exact colors from NB
    \definecolor{incolor}{rgb}{0.0, 0.0, 0.5}
    \definecolor{outcolor}{rgb}{0.545, 0.0, 0.0}



    
    % Prevent overflowing lines due to hard-to-break entities
    \sloppy 
    % Setup hyperref package
    \hypersetup{
      breaklinks=true,  % so long urls are correctly broken across lines
      colorlinks=true,
      urlcolor=urlcolor,
      linkcolor=linkcolor,
      citecolor=citecolor,
      }
    % Slightly bigger margins than the latex defaults
    
    \geometry{verbose,tmargin=1in,bmargin=1in,lmargin=1in,rmargin=1in}
    
    

    \begin{document}
    
    
    \maketitle
    
    

    
    \section{Classifying Urban sounds using Deep
Learning}\label{classifying-urban-sounds-using-deep-learning}

\subsection{1 Data Exploration and
Visualisation}\label{data-exploration-and-visualisation}

    \subsubsection{UrbanSound dataset}\label{urbansound-dataset}

I will use a dataset called Urbansound8K for this project. The dataset
contains 8732 sound excerpts (\textless{}=4s) of urban sounds in10
classes, which are:

\begin{verbatim}
<li>Air Conditioner</li>
<li>Car Horn</li>
<li>Children Playing</li>
<li>Dog bark</li>
<li>Drilling</li>
<li>Engine Idling</li>
<li>Gun Shot</li>
<li>Jackhammer</li>
<li>Siren</li>
<li>Street Music</li>
\end{verbatim}

This metadata contains a unique ID for each sound excerpt along with
it's given class name.

A sample of this dataset is included with the accompanying git repo and
the full dataset can be access from
\href{https://urbansounddataset.weebly.com/urbansound8k.html}{here}.

    \subsubsection{Audio sample file data
overview}\label{audio-sample-file-data-overview}

Used sound are digital audio files in .wav format.

Sound waves are digitised by sampling them at discrete intervals known
as the sampling rate.

The bit depth determines how detailed the sample will be also known as
the dynamic range of the signal (typically 16bit which means a sample
can range from 65,536 amplitude values).

Therefore, the data we will be analysing for each sound excerpts is
essentially a one dimensional array or vector of amplitude values.

    \subsubsection{Analysing audio data}\label{analysing-audio-data}

For audio analysis, we will be using the following libraries:

\paragraph{1. IPython.display.Audio}\label{ipython.display.audio}

This allows us to play audio directly in the Jupyter Notebook.

\paragraph{2. Librosa}\label{librosa}

librosa is a Python package for music and audio processing by Brian
McFee and will allow us to load audio in our notebook as a numpy array
for analysis and manipulation.

You may need to install librosa using pip as follows:

\texttt{pip\ install\ librosa}

    \subsubsection{Auditory inspection}\label{auditory-inspection}

I will use \texttt{IPython.display.Audio} to play the audio files so we
can inspect aurally.

    \begin{Verbatim}[commandchars=\\\{\}]
{\color{incolor}In [{\color{incolor}1}]:} \PY{k+kn}{import} \PY{n+nn}{IPython}\PY{n+nn}{.}\PY{n+nn}{display} \PY{k}{as} \PY{n+nn}{ipd}
        
        \PY{n}{ipd}\PY{o}{.}\PY{n}{Audio}\PY{p}{(}\PY{l+s+s1}{\PYZsq{}}\PY{l+s+s1}{../UrbanSound Dataset sample/audio/100032\PYZhy{}3\PYZhy{}0\PYZhy{}0.wav}\PY{l+s+s1}{\PYZsq{}}\PY{p}{)}
\end{Verbatim}


\begin{Verbatim}[commandchars=\\\{\}]
{\color{outcolor}Out[{\color{outcolor}1}]:} <IPython.lib.display.Audio object>
\end{Verbatim}
            
    \subsubsection{Visual inspection}\label{visual-inspection}

I will load a sample from each class and visually inspect the data for
any patterns. I will use librosa to load the audio file into an array
then librosa.display and matplotlib to display the waveform.

    \begin{Verbatim}[commandchars=\\\{\}]
{\color{incolor}In [{\color{incolor}2}]:} \PY{c+c1}{\PYZsh{} Load imports}
        
        \PY{k+kn}{import} \PY{n+nn}{IPython}\PY{n+nn}{.}\PY{n+nn}{display} \PY{k}{as} \PY{n+nn}{ipd}
        \PY{k+kn}{import} \PY{n+nn}{librosa}
        \PY{k+kn}{import} \PY{n+nn}{librosa}\PY{n+nn}{.}\PY{n+nn}{display}
        \PY{k+kn}{import} \PY{n+nn}{matplotlib}\PY{n+nn}{.}\PY{n+nn}{pyplot} \PY{k}{as} \PY{n+nn}{plt}
\end{Verbatim}


    \begin{Verbatim}[commandchars=\\\{\}]
{\color{incolor}In [{\color{incolor}3}]:} \PY{c+c1}{\PYZsh{} Class: Air Conditioner}
        
        \PY{n}{filename} \PY{o}{=} \PY{l+s+s1}{\PYZsq{}}\PY{l+s+s1}{../UrbanSound Dataset sample/audio/100852\PYZhy{}0\PYZhy{}0\PYZhy{}0.wav}\PY{l+s+s1}{\PYZsq{}}
        \PY{n}{plt}\PY{o}{.}\PY{n}{figure}\PY{p}{(}\PY{n}{figsize}\PY{o}{=}\PY{p}{(}\PY{l+m+mi}{12}\PY{p}{,}\PY{l+m+mi}{4}\PY{p}{)}\PY{p}{)}
        \PY{n}{data}\PY{p}{,}\PY{n}{sample\PYZus{}rate} \PY{o}{=} \PY{n}{librosa}\PY{o}{.}\PY{n}{load}\PY{p}{(}\PY{n}{filename}\PY{p}{)}
        \PY{n}{\PYZus{}} \PY{o}{=} \PY{n}{librosa}\PY{o}{.}\PY{n}{display}\PY{o}{.}\PY{n}{waveplot}\PY{p}{(}\PY{n}{data}\PY{p}{,}\PY{n}{sr}\PY{o}{=}\PY{n}{sample\PYZus{}rate}\PY{p}{)}
        \PY{n}{ipd}\PY{o}{.}\PY{n}{Audio}\PY{p}{(}\PY{n}{filename}\PY{p}{)}
\end{Verbatim}


\begin{Verbatim}[commandchars=\\\{\}]
{\color{outcolor}Out[{\color{outcolor}3}]:} <IPython.lib.display.Audio object>
\end{Verbatim}
            
    \begin{center}
    \adjustimage{max size={0.9\linewidth}{0.9\paperheight}}{output_8_1.png}
    \end{center}
    { \hspace*{\fill} \\}
    
    \begin{Verbatim}[commandchars=\\\{\}]
{\color{incolor}In [{\color{incolor}4}]:} \PY{c+c1}{\PYZsh{} Class: Car horn }
        
        \PY{n}{filename} \PY{o}{=} \PY{l+s+s1}{\PYZsq{}}\PY{l+s+s1}{../UrbanSound Dataset sample/audio/100648\PYZhy{}1\PYZhy{}0\PYZhy{}0.wav}\PY{l+s+s1}{\PYZsq{}}
        \PY{n}{plt}\PY{o}{.}\PY{n}{figure}\PY{p}{(}\PY{n}{figsize}\PY{o}{=}\PY{p}{(}\PY{l+m+mi}{12}\PY{p}{,}\PY{l+m+mi}{4}\PY{p}{)}\PY{p}{)}
        \PY{n}{data}\PY{p}{,}\PY{n}{sample\PYZus{}rate} \PY{o}{=} \PY{n}{librosa}\PY{o}{.}\PY{n}{load}\PY{p}{(}\PY{n}{filename}\PY{p}{)}
        \PY{n}{\PYZus{}} \PY{o}{=} \PY{n}{librosa}\PY{o}{.}\PY{n}{display}\PY{o}{.}\PY{n}{waveplot}\PY{p}{(}\PY{n}{data}\PY{p}{,}\PY{n}{sr}\PY{o}{=}\PY{n}{sample\PYZus{}rate}\PY{p}{)}
        \PY{n}{ipd}\PY{o}{.}\PY{n}{Audio}\PY{p}{(}\PY{n}{filename}\PY{p}{)}
\end{Verbatim}


\begin{Verbatim}[commandchars=\\\{\}]
{\color{outcolor}Out[{\color{outcolor}4}]:} <IPython.lib.display.Audio object>
\end{Verbatim}
            
    \begin{center}
    \adjustimage{max size={0.9\linewidth}{0.9\paperheight}}{output_9_1.png}
    \end{center}
    { \hspace*{\fill} \\}
    
    \begin{Verbatim}[commandchars=\\\{\}]
{\color{incolor}In [{\color{incolor}5}]:} \PY{c+c1}{\PYZsh{} Class: Children playing }
        
        \PY{n}{filename} \PY{o}{=} \PY{l+s+s1}{\PYZsq{}}\PY{l+s+s1}{../UrbanSound Dataset sample/audio/100263\PYZhy{}2\PYZhy{}0\PYZhy{}117.wav}\PY{l+s+s1}{\PYZsq{}}
        \PY{n}{plt}\PY{o}{.}\PY{n}{figure}\PY{p}{(}\PY{n}{figsize}\PY{o}{=}\PY{p}{(}\PY{l+m+mi}{12}\PY{p}{,}\PY{l+m+mi}{4}\PY{p}{)}\PY{p}{)}
        \PY{n}{data}\PY{p}{,}\PY{n}{sample\PYZus{}rate} \PY{o}{=} \PY{n}{librosa}\PY{o}{.}\PY{n}{load}\PY{p}{(}\PY{n}{filename}\PY{p}{)}
        \PY{n}{\PYZus{}} \PY{o}{=} \PY{n}{librosa}\PY{o}{.}\PY{n}{display}\PY{o}{.}\PY{n}{waveplot}\PY{p}{(}\PY{n}{data}\PY{p}{,}\PY{n}{sr}\PY{o}{=}\PY{n}{sample\PYZus{}rate}\PY{p}{)}
        \PY{n}{ipd}\PY{o}{.}\PY{n}{Audio}\PY{p}{(}\PY{n}{filename}\PY{p}{)}
\end{Verbatim}


\begin{Verbatim}[commandchars=\\\{\}]
{\color{outcolor}Out[{\color{outcolor}5}]:} <IPython.lib.display.Audio object>
\end{Verbatim}
            
    \begin{center}
    \adjustimage{max size={0.9\linewidth}{0.9\paperheight}}{output_10_1.png}
    \end{center}
    { \hspace*{\fill} \\}
    
    \begin{Verbatim}[commandchars=\\\{\}]
{\color{incolor}In [{\color{incolor}6}]:} \PY{c+c1}{\PYZsh{} Class: Dog bark}
        
        \PY{n}{filename} \PY{o}{=} \PY{l+s+s1}{\PYZsq{}}\PY{l+s+s1}{../UrbanSound Dataset sample/audio/100032\PYZhy{}3\PYZhy{}0\PYZhy{}0.wav}\PY{l+s+s1}{\PYZsq{}}
        \PY{n}{plt}\PY{o}{.}\PY{n}{figure}\PY{p}{(}\PY{n}{figsize}\PY{o}{=}\PY{p}{(}\PY{l+m+mi}{12}\PY{p}{,}\PY{l+m+mi}{4}\PY{p}{)}\PY{p}{)}
        \PY{n}{data}\PY{p}{,}\PY{n}{sample\PYZus{}rate} \PY{o}{=} \PY{n}{librosa}\PY{o}{.}\PY{n}{load}\PY{p}{(}\PY{n}{filename}\PY{p}{)}
        \PY{n}{\PYZus{}} \PY{o}{=} \PY{n}{librosa}\PY{o}{.}\PY{n}{display}\PY{o}{.}\PY{n}{waveplot}\PY{p}{(}\PY{n}{data}\PY{p}{,}\PY{n}{sr}\PY{o}{=}\PY{n}{sample\PYZus{}rate}\PY{p}{)}
        \PY{n}{ipd}\PY{o}{.}\PY{n}{Audio}\PY{p}{(}\PY{n}{filename}\PY{p}{)}
\end{Verbatim}


\begin{Verbatim}[commandchars=\\\{\}]
{\color{outcolor}Out[{\color{outcolor}6}]:} <IPython.lib.display.Audio object>
\end{Verbatim}
            
    \begin{center}
    \adjustimage{max size={0.9\linewidth}{0.9\paperheight}}{output_11_1.png}
    \end{center}
    { \hspace*{\fill} \\}
    
    \begin{Verbatim}[commandchars=\\\{\}]
{\color{incolor}In [{\color{incolor}7}]:} \PY{c+c1}{\PYZsh{} Class: Drilling}
        
        \PY{n}{filename} \PY{o}{=} \PY{l+s+s1}{\PYZsq{}}\PY{l+s+s1}{../UrbanSound Dataset sample/audio/103199\PYZhy{}4\PYZhy{}0\PYZhy{}0.wav}\PY{l+s+s1}{\PYZsq{}}
        \PY{n}{plt}\PY{o}{.}\PY{n}{figure}\PY{p}{(}\PY{n}{figsize}\PY{o}{=}\PY{p}{(}\PY{l+m+mi}{12}\PY{p}{,}\PY{l+m+mi}{4}\PY{p}{)}\PY{p}{)}
        \PY{n}{data}\PY{p}{,}\PY{n}{sample\PYZus{}rate} \PY{o}{=} \PY{n}{librosa}\PY{o}{.}\PY{n}{load}\PY{p}{(}\PY{n}{filename}\PY{p}{)}
        \PY{n}{\PYZus{}} \PY{o}{=} \PY{n}{librosa}\PY{o}{.}\PY{n}{display}\PY{o}{.}\PY{n}{waveplot}\PY{p}{(}\PY{n}{data}\PY{p}{,}\PY{n}{sr}\PY{o}{=}\PY{n}{sample\PYZus{}rate}\PY{p}{)}
        \PY{n}{ipd}\PY{o}{.}\PY{n}{Audio}\PY{p}{(}\PY{n}{filename}\PY{p}{)}
\end{Verbatim}


\begin{Verbatim}[commandchars=\\\{\}]
{\color{outcolor}Out[{\color{outcolor}7}]:} <IPython.lib.display.Audio object>
\end{Verbatim}
            
    \begin{center}
    \adjustimage{max size={0.9\linewidth}{0.9\paperheight}}{output_12_1.png}
    \end{center}
    { \hspace*{\fill} \\}
    
    \begin{Verbatim}[commandchars=\\\{\}]
{\color{incolor}In [{\color{incolor}8}]:} \PY{c+c1}{\PYZsh{} Class: Engine Idling }
        
        \PY{n}{filename} \PY{o}{=} \PY{l+s+s1}{\PYZsq{}}\PY{l+s+s1}{../UrbanSound Dataset sample/audio/102857\PYZhy{}5\PYZhy{}0\PYZhy{}0.wav}\PY{l+s+s1}{\PYZsq{}}
        \PY{n}{plt}\PY{o}{.}\PY{n}{figure}\PY{p}{(}\PY{n}{figsize}\PY{o}{=}\PY{p}{(}\PY{l+m+mi}{12}\PY{p}{,}\PY{l+m+mi}{4}\PY{p}{)}\PY{p}{)}
        \PY{n}{data}\PY{p}{,}\PY{n}{sample\PYZus{}rate} \PY{o}{=} \PY{n}{librosa}\PY{o}{.}\PY{n}{load}\PY{p}{(}\PY{n}{filename}\PY{p}{)}
        \PY{n}{\PYZus{}} \PY{o}{=} \PY{n}{librosa}\PY{o}{.}\PY{n}{display}\PY{o}{.}\PY{n}{waveplot}\PY{p}{(}\PY{n}{data}\PY{p}{,}\PY{n}{sr}\PY{o}{=}\PY{n}{sample\PYZus{}rate}\PY{p}{)}
        \PY{n}{ipd}\PY{o}{.}\PY{n}{Audio}\PY{p}{(}\PY{n}{filename}\PY{p}{)}
\end{Verbatim}


\begin{Verbatim}[commandchars=\\\{\}]
{\color{outcolor}Out[{\color{outcolor}8}]:} <IPython.lib.display.Audio object>
\end{Verbatim}
            
    \begin{center}
    \adjustimage{max size={0.9\linewidth}{0.9\paperheight}}{output_13_1.png}
    \end{center}
    { \hspace*{\fill} \\}
    
    \begin{Verbatim}[commandchars=\\\{\}]
{\color{incolor}In [{\color{incolor}9}]:} \PY{c+c1}{\PYZsh{} Class: Gunshot}
        
        \PY{n}{filename} \PY{o}{=} \PY{l+s+s1}{\PYZsq{}}\PY{l+s+s1}{../UrbanSound Dataset sample/audio/102305\PYZhy{}6\PYZhy{}0\PYZhy{}0.wav}\PY{l+s+s1}{\PYZsq{}}
        \PY{n}{plt}\PY{o}{.}\PY{n}{figure}\PY{p}{(}\PY{n}{figsize}\PY{o}{=}\PY{p}{(}\PY{l+m+mi}{12}\PY{p}{,}\PY{l+m+mi}{4}\PY{p}{)}\PY{p}{)}
        \PY{n}{data}\PY{p}{,}\PY{n}{sample\PYZus{}rate} \PY{o}{=} \PY{n}{librosa}\PY{o}{.}\PY{n}{load}\PY{p}{(}\PY{n}{filename}\PY{p}{)}
        \PY{n}{\PYZus{}} \PY{o}{=} \PY{n}{librosa}\PY{o}{.}\PY{n}{display}\PY{o}{.}\PY{n}{waveplot}\PY{p}{(}\PY{n}{data}\PY{p}{,}\PY{n}{sr}\PY{o}{=}\PY{n}{sample\PYZus{}rate}\PY{p}{)}
        \PY{n}{ipd}\PY{o}{.}\PY{n}{Audio}\PY{p}{(}\PY{n}{filename}\PY{p}{)}
\end{Verbatim}


\begin{Verbatim}[commandchars=\\\{\}]
{\color{outcolor}Out[{\color{outcolor}9}]:} <IPython.lib.display.Audio object>
\end{Verbatim}
            
    \begin{center}
    \adjustimage{max size={0.9\linewidth}{0.9\paperheight}}{output_14_1.png}
    \end{center}
    { \hspace*{\fill} \\}
    
    \begin{Verbatim}[commandchars=\\\{\}]
{\color{incolor}In [{\color{incolor}10}]:} \PY{c+c1}{\PYZsh{} Class: Jackhammer}
         
         \PY{n}{filename} \PY{o}{=} \PY{l+s+s1}{\PYZsq{}}\PY{l+s+s1}{../UrbanSound Dataset sample/audio/103074\PYZhy{}7\PYZhy{}0\PYZhy{}0.wav}\PY{l+s+s1}{\PYZsq{}}
         \PY{n}{plt}\PY{o}{.}\PY{n}{figure}\PY{p}{(}\PY{n}{figsize}\PY{o}{=}\PY{p}{(}\PY{l+m+mi}{12}\PY{p}{,}\PY{l+m+mi}{4}\PY{p}{)}\PY{p}{)}
         \PY{n}{data}\PY{p}{,}\PY{n}{sample\PYZus{}rate} \PY{o}{=} \PY{n}{librosa}\PY{o}{.}\PY{n}{load}\PY{p}{(}\PY{n}{filename}\PY{p}{)}
         \PY{n}{\PYZus{}} \PY{o}{=} \PY{n}{librosa}\PY{o}{.}\PY{n}{display}\PY{o}{.}\PY{n}{waveplot}\PY{p}{(}\PY{n}{data}\PY{p}{,}\PY{n}{sr}\PY{o}{=}\PY{n}{sample\PYZus{}rate}\PY{p}{)}
         \PY{n}{ipd}\PY{o}{.}\PY{n}{Audio}\PY{p}{(}\PY{n}{filename}\PY{p}{)}
\end{Verbatim}


\begin{Verbatim}[commandchars=\\\{\}]
{\color{outcolor}Out[{\color{outcolor}10}]:} <IPython.lib.display.Audio object>
\end{Verbatim}
            
    \begin{center}
    \adjustimage{max size={0.9\linewidth}{0.9\paperheight}}{output_15_1.png}
    \end{center}
    { \hspace*{\fill} \\}
    
    \begin{Verbatim}[commandchars=\\\{\}]
{\color{incolor}In [{\color{incolor}11}]:} \PY{c+c1}{\PYZsh{} Class: Siren}
         
         \PY{n}{filename} \PY{o}{=} \PY{l+s+s1}{\PYZsq{}}\PY{l+s+s1}{../UrbanSound Dataset sample/audio/102853\PYZhy{}8\PYZhy{}0\PYZhy{}0.wav}\PY{l+s+s1}{\PYZsq{}}
         \PY{n}{plt}\PY{o}{.}\PY{n}{figure}\PY{p}{(}\PY{n}{figsize}\PY{o}{=}\PY{p}{(}\PY{l+m+mi}{12}\PY{p}{,}\PY{l+m+mi}{4}\PY{p}{)}\PY{p}{)}
         \PY{n}{data}\PY{p}{,}\PY{n}{sample\PYZus{}rate} \PY{o}{=} \PY{n}{librosa}\PY{o}{.}\PY{n}{load}\PY{p}{(}\PY{n}{filename}\PY{p}{)}
         \PY{n}{\PYZus{}} \PY{o}{=} \PY{n}{librosa}\PY{o}{.}\PY{n}{display}\PY{o}{.}\PY{n}{waveplot}\PY{p}{(}\PY{n}{data}\PY{p}{,}\PY{n}{sr}\PY{o}{=}\PY{n}{sample\PYZus{}rate}\PY{p}{)}
         \PY{n}{ipd}\PY{o}{.}\PY{n}{Audio}\PY{p}{(}\PY{n}{filename}\PY{p}{)}
\end{Verbatim}


\begin{Verbatim}[commandchars=\\\{\}]
{\color{outcolor}Out[{\color{outcolor}11}]:} <IPython.lib.display.Audio object>
\end{Verbatim}
            
    \begin{center}
    \adjustimage{max size={0.9\linewidth}{0.9\paperheight}}{output_16_1.png}
    \end{center}
    { \hspace*{\fill} \\}
    
    \begin{Verbatim}[commandchars=\\\{\}]
{\color{incolor}In [{\color{incolor}12}]:} \PY{c+c1}{\PYZsh{} Class: Street music}
         
         \PY{n}{filename} \PY{o}{=} \PY{l+s+s1}{\PYZsq{}}\PY{l+s+s1}{../UrbanSound Dataset sample/audio/101848\PYZhy{}9\PYZhy{}0\PYZhy{}0.wav}\PY{l+s+s1}{\PYZsq{}}
         \PY{n}{plt}\PY{o}{.}\PY{n}{figure}\PY{p}{(}\PY{n}{figsize}\PY{o}{=}\PY{p}{(}\PY{l+m+mi}{12}\PY{p}{,}\PY{l+m+mi}{4}\PY{p}{)}\PY{p}{)}
         \PY{n}{data}\PY{p}{,}\PY{n}{sample\PYZus{}rate} \PY{o}{=} \PY{n}{librosa}\PY{o}{.}\PY{n}{load}\PY{p}{(}\PY{n}{filename}\PY{p}{)}
         \PY{n}{\PYZus{}} \PY{o}{=} \PY{n}{librosa}\PY{o}{.}\PY{n}{display}\PY{o}{.}\PY{n}{waveplot}\PY{p}{(}\PY{n}{data}\PY{p}{,}\PY{n}{sr}\PY{o}{=}\PY{n}{sample\PYZus{}rate}\PY{p}{)}
         \PY{n}{ipd}\PY{o}{.}\PY{n}{Audio}\PY{p}{(}\PY{n}{filename}\PY{p}{)}
\end{Verbatim}


\begin{Verbatim}[commandchars=\\\{\}]
{\color{outcolor}Out[{\color{outcolor}12}]:} <IPython.lib.display.Audio object>
\end{Verbatim}
            
    \begin{center}
    \adjustimage{max size={0.9\linewidth}{0.9\paperheight}}{output_17_1.png}
    \end{center}
    { \hspace*{\fill} \\}
    
    \subsubsection{Observations}\label{observations}

From a visual inspection we can see that it is tricky to visualise the
difference between some of the classes.

Particularly, the waveforms for reptitive sounds for air conditioner,
drilling, engine idling and jackhammer are similar in shape.

Likewise the peak in the dog barking sample is simmilar in shape to the
gun shot sample (albeit the samples differ in that there are two peaks
for two gunshots compared to the one peak for one dog bark). Also, the
car horn is similar too.

We show to similarities between the children playing and street music.

The human ear can naturally detect the difference between the harmonics,
it will be interesting to see how well a deep learning model will be
able to extract the necessary features to distinguish between these
classes.

However, it is easy to differentiate from the waveform shape, the
difference between certain classes such as dog barking and engine
idling.

    \subsubsection{Dataset Metadata}\label{dataset-metadata}

Here we will load the UrbanSound metadata .csv file into a Panda
dataframe.

    \begin{Verbatim}[commandchars=\\\{\}]
{\color{incolor}In [{\color{incolor}13}]:} \PY{k+kn}{import} \PY{n+nn}{pandas} \PY{k}{as} \PY{n+nn}{pd}
         \PY{n}{metadata} \PY{o}{=} \PY{n}{pd}\PY{o}{.}\PY{n}{read\PYZus{}csv}\PY{p}{(}\PY{l+s+s1}{\PYZsq{}}\PY{l+s+s1}{../UrbanSound Dataset sample/metadata/UrbanSound8K.csv}\PY{l+s+s1}{\PYZsq{}}\PY{p}{)}
         \PY{n}{metadata}\PY{o}{.}\PY{n}{head}\PY{p}{(}\PY{p}{)}
\end{Verbatim}


\begin{Verbatim}[commandchars=\\\{\}]
{\color{outcolor}Out[{\color{outcolor}13}]:}       slice\_file\_name    fsID  start        end  salience  fold  classID  \textbackslash{}
         0    100032-3-0-0.wav  100032    0.0   0.317551         1     5        3   
         1  100263-2-0-117.wav  100263   58.5  62.500000         1     5        2   
         2  100263-2-0-121.wav  100263   60.5  64.500000         1     5        2   
         3  100263-2-0-126.wav  100263   63.0  67.000000         1     5        2   
         4  100263-2-0-137.wav  100263   68.5  72.500000         1     5        2   
         
                  class\_name  
         0          dog\_bark  
         1  children\_playing  
         2  children\_playing  
         3  children\_playing  
         4  children\_playing  
\end{Verbatim}
            
    \subsubsection{Class distributions}\label{class-distributions}

    \begin{Verbatim}[commandchars=\\\{\}]
{\color{incolor}In [{\color{incolor}14}]:} \PY{n+nb}{print}\PY{p}{(}\PY{n}{metadata}\PY{o}{.}\PY{n}{class\PYZus{}name}\PY{o}{.}\PY{n}{value\PYZus{}counts}\PY{p}{(}\PY{p}{)}\PY{p}{)}
\end{Verbatim}


    \begin{Verbatim}[commandchars=\\\{\}]
children\_playing    1000
dog\_bark            1000
street\_music        1000
jackhammer          1000
engine\_idling       1000
air\_conditioner     1000
drilling            1000
siren                929
car\_horn             429
gun\_shot             374
Name: class\_name, dtype: int64

    \end{Verbatim}

    \subsubsection{Observations}\label{observations}

Here I can see the Class labels are unbalanced. Although 7 out of the 10
classes all have exactly 1000 samples, and siren is not far off with
929, the remaining two (car\_horn, gun\_shot) have significantly less
samples at 43\% and 37\% respectively.

This will be a concern and something we may need to address later on.

    \subsubsection{Audio sample file
properties}\label{audio-sample-file-properties}

Next I will iterate through each of the audio sample files and extract,
number of audio channels, sample rate and bit-depth.

    \begin{Verbatim}[commandchars=\\\{\}]
{\color{incolor}In [{\color{incolor}15}]:} \PY{c+c1}{\PYZsh{} Load various imports }
         \PY{k+kn}{import} \PY{n+nn}{pandas} \PY{k}{as} \PY{n+nn}{pd}
         \PY{k+kn}{import} \PY{n+nn}{os}
         \PY{k+kn}{import} \PY{n+nn}{librosa}
         \PY{k+kn}{import} \PY{n+nn}{librosa}\PY{n+nn}{.}\PY{n+nn}{display}
         
         \PY{k+kn}{from} \PY{n+nn}{helpers}\PY{n+nn}{.}\PY{n+nn}{wavfilehelper} \PY{k}{import} \PY{n}{WavFileHelper}
         \PY{n}{wavfilehelper} \PY{o}{=} \PY{n}{WavFileHelper}\PY{p}{(}\PY{p}{)}
         
         \PY{n}{audiodata} \PY{o}{=} \PY{p}{[}\PY{p}{]}
         \PY{k}{for} \PY{n}{index}\PY{p}{,} \PY{n}{row} \PY{o+ow}{in} \PY{n}{metadata}\PY{o}{.}\PY{n}{iterrows}\PY{p}{(}\PY{p}{)}\PY{p}{:}
             
             \PY{n}{file\PYZus{}name} \PY{o}{=} \PY{n}{os}\PY{o}{.}\PY{n}{path}\PY{o}{.}\PY{n}{join}\PY{p}{(}\PY{n}{os}\PY{o}{.}\PY{n}{path}\PY{o}{.}\PY{n}{abspath}\PY{p}{(}\PY{l+s+s1}{\PYZsq{}}\PY{l+s+s1}{/Volumes/Untitled/ML\PYZus{}Data/Urban Sound/UrbanSound8K/audio/}\PY{l+s+s1}{\PYZsq{}}\PY{p}{)}\PY{p}{,}\PY{l+s+s1}{\PYZsq{}}\PY{l+s+s1}{fold}\PY{l+s+s1}{\PYZsq{}}\PY{o}{+}\PY{n+nb}{str}\PY{p}{(}\PY{n}{row}\PY{p}{[}\PY{l+s+s2}{\PYZdq{}}\PY{l+s+s2}{fold}\PY{l+s+s2}{\PYZdq{}}\PY{p}{]}\PY{p}{)}\PY{o}{+}\PY{l+s+s1}{\PYZsq{}}\PY{l+s+s1}{/}\PY{l+s+s1}{\PYZsq{}}\PY{p}{,}\PY{n+nb}{str}\PY{p}{(}\PY{n}{row}\PY{p}{[}\PY{l+s+s2}{\PYZdq{}}\PY{l+s+s2}{slice\PYZus{}file\PYZus{}name}\PY{l+s+s2}{\PYZdq{}}\PY{p}{]}\PY{p}{)}\PY{p}{)}
             \PY{n}{data} \PY{o}{=} \PY{n}{wavfilehelper}\PY{o}{.}\PY{n}{read\PYZus{}file\PYZus{}properties}\PY{p}{(}\PY{n}{file\PYZus{}name}\PY{p}{)}
             \PY{n}{audiodata}\PY{o}{.}\PY{n}{append}\PY{p}{(}\PY{n}{data}\PY{p}{)}
         
         \PY{c+c1}{\PYZsh{} Convert into a Panda dataframe}
         \PY{n}{audiodf} \PY{o}{=} \PY{n}{pd}\PY{o}{.}\PY{n}{DataFrame}\PY{p}{(}\PY{n}{audiodata}\PY{p}{,} \PY{n}{columns}\PY{o}{=}\PY{p}{[}\PY{l+s+s1}{\PYZsq{}}\PY{l+s+s1}{num\PYZus{}channels}\PY{l+s+s1}{\PYZsq{}}\PY{p}{,}\PY{l+s+s1}{\PYZsq{}}\PY{l+s+s1}{sample\PYZus{}rate}\PY{l+s+s1}{\PYZsq{}}\PY{p}{,}\PY{l+s+s1}{\PYZsq{}}\PY{l+s+s1}{bit\PYZus{}depth}\PY{l+s+s1}{\PYZsq{}}\PY{p}{]}\PY{p}{)}
\end{Verbatim}


    \subsubsection{Audio channels}\label{audio-channels}

Most of the samples have two audio channels (meaning stereo) with a few
with just the one channel (mono).

The easiest option here to make them uniform will be to merge the two
channels in the stero samples into one by averaging the values of the
two channels.

    \begin{Verbatim}[commandchars=\\\{\}]
{\color{incolor}In [{\color{incolor}19}]:} \PY{c+c1}{\PYZsh{} num of channels }
         
         \PY{n+nb}{print}\PY{p}{(}\PY{n}{audiodf}\PY{o}{.}\PY{n}{num\PYZus{}channels}\PY{o}{.}\PY{n}{value\PYZus{}counts}\PY{p}{(}\PY{n}{normalize}\PY{o}{=}\PY{k+kc}{True}\PY{p}{)}\PY{p}{)}
\end{Verbatim}


    \begin{Verbatim}[commandchars=\\\{\}]
2    0.915369
1    0.084631
Name: num\_channels, dtype: float64

    \end{Verbatim}

    \subsubsection{Sample rate}\label{sample-rate}

There is a wide range of Sample rates that have been used across all the
samples which is a concern (ranging from 96k to 8k).

This likley means that we will have to apply a sample-rate conversion
technique (either up-conversion or down-conversion) so we can see an
agnostic representation of their waveform which will allow us to do a
fair comparison.

    \begin{Verbatim}[commandchars=\\\{\}]
{\color{incolor}In [{\color{incolor}21}]:} \PY{c+c1}{\PYZsh{} sample rates }
         
         \PY{n+nb}{print}\PY{p}{(}\PY{n}{audiodf}\PY{o}{.}\PY{n}{sample\PYZus{}rate}\PY{o}{.}\PY{n}{value\PYZus{}counts}\PY{p}{(}\PY{n}{normalize}\PY{o}{=}\PY{k+kc}{True}\PY{p}{)}\PY{p}{)}
\end{Verbatim}


    \begin{Verbatim}[commandchars=\\\{\}]
44100     0.614979
48000     0.286532
96000     0.069858
24000     0.009391
16000     0.005153
22050     0.005039
11025     0.004466
192000    0.001947
8000      0.001374
11024     0.000802
32000     0.000458
Name: sample\_rate, dtype: float64

    \end{Verbatim}

    \subsubsection{Bit-depth}\label{bit-depth}

There is also a wide range of bit-depths. It's likely that we may need
to normalise them by taking the maximum and minimum amplitude values for
a given bit-depth.

    \begin{Verbatim}[commandchars=\\\{\}]
{\color{incolor}In [{\color{incolor}22}]:} \PY{c+c1}{\PYZsh{} bit depth}
         
         \PY{n+nb}{print}\PY{p}{(}\PY{n}{audiodf}\PY{o}{.}\PY{n}{bit\PYZus{}depth}\PY{o}{.}\PY{n}{value\PYZus{}counts}\PY{p}{(}\PY{n}{normalize}\PY{o}{=}\PY{k+kc}{True}\PY{p}{)}\PY{p}{)}
\end{Verbatim}


    \begin{Verbatim}[commandchars=\\\{\}]
16    0.659414
24    0.315277
32    0.019354
8     0.004924
4     0.001031
Name: bit\_depth, dtype: float64

    \end{Verbatim}

    \subsubsection{Other audio properties to
consider}\label{other-audio-properties-to-consider}

I may also need to consider normalising the volume levels (wave
amplitude value) if this is seen to vary greatly, by either looking at
the peak volume or the RMS volume.

    \subsubsection{\texorpdfstring{\emph{In the next notebook we will
preprocess the
data}}{In the next notebook we will preprocess the data}}\label{in-the-next-notebook-we-will-preprocess-the-data}


    % Add a bibliography block to the postdoc
    
    
    
    \end{document}
